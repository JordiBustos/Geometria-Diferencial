\documentclass{article}

\usepackage[margin=1in]{geometry} 
\usepackage{amsmath,amsthm,amssymb}
\usepackage[spanish]{babel}
\usepackage{newpxtext,newpxmath}

\newcommand{\R}{\mathbb{R}}
\newcommand{\Z}{\mathbb{Z}}
\newcommand{\N}{\mathbb{N}}
\newcommand{\Q}{\mathbb{Q}}
\newcommand{\C}{\mathbb{C}}

\newenvironment{theorem}[2][Ejercicio]{\begin{trivlist}
\item[\hskip \labelsep {\bfseries #1}\hskip \labelsep {\bfseries #2.}]}{\end{trivlist}}
\newenvironment{lemma}[2][Lemma]{\begin{trivlist}
\item[\hskip \labelsep {\bfseries #1}\hskip \labelsep {\bfseries #2.}]}{\end{trivlist}}
\newenvironment{exercise}[2][Exercise]{\begin{trivlist}
\item[\hskip \labelsep {\bfseries #1}\hskip \labelsep {\bfseries #2.}]}{\end{trivlist}}
\newenvironment{problem}[2][Problem]{\begin{trivlist}
\item[\hskip \labelsep {\bfseries #1}\hskip \labelsep {\bfseries #2.}]}{\end{trivlist}}
\newenvironment{question}[2][Question]{\begin{trivlist}
\item[\hskip \labelsep {\bfseries #1}\hskip \labelsep {\bfseries #2.}]}{\end{trivlist}}
\newenvironment{corollary}[2][Corollary]{\begin{trivlist}
\item[\hskip \labelsep {\bfseries #1}\hskip \labelsep {\bfseries #2.}]}{\end{trivlist}}

\newenvironment{solution}{\begin{proof}[Solution]}{\end{proof}}

\begin{document}

\title{Geometría Diferencial}
\author{Bustos Jordi\\Práctica V}

\maketitle

\begin{theorem}{1}
    Consideremos la tractriz dada por \( \gamma(t) = (sen(t), 0, \ln(\tan(t/2))) \), con \( t \in (\pi/2, \pi) \). La superficie de revolución obtenida al rotar la tractriz alrededor del eje \( z \) es llamada
    pseudoesfera. Probar que la curvatura de Gauss de la pseudoesfera es \( K \equiv -1 \).
\end{theorem}

\begin{proof}
    La parametrización de la pseudoesfera es \begin{align*}
        \varphi(u, v) & = (f(v) \cos(u), f(v) \sin(u), g(v))                                                               \\
                      & = (\sen(v) \cos(u), \sen(v) \sin(u), \ln(\tan(v/2))), \quad (u \in [0, 2\pi), v \in (\pi/2, \pi)).
    \end{align*}
    Luego, por \textbf{Ejemplo 4.8.4} se tiene que \begin{align*}
        K & = -\frac{f''(v)}{f(v)} = -\frac{-\sen(v)}{\sen(v)} = -1.
    \end{align*}
    Que es lo que queríamos demostrar.
\end{proof}

\vspace{1em}

\begin{theorem}{2}
    Consideremos el hiperboloide de la ecuación \( z = axy \) con \( a \in \R \). Mostrar que en el origen la curvatura de Gauss es \( K = -a^2 \) y la curvatura media es \( H = 0 \).
\end{theorem}

\begin{proof}
    La parametrización del hiperboloide es \begin{align*}
        \varphi(u, v) & = (u, v, auv), \quad (u, v \in \R).
    \end{align*}
    Luego, \begin{align*}
        \varphi_u & = (1, 0, av),                                                                                                           \\
        \varphi_v & = (0, 1, au),                                                                                                           \\
        N         & = \frac{\varphi_u \times \varphi_v}{\|\varphi_u \times \varphi_v\|} = \frac{(-av, -au, 1)}{\sqrt{1 + a^2u^2 + a^2v^2}}.
    \end{align*}
    Luego, la aplicación de Weingarten viene dada por: \begin{align*}
        W_p(\varphi_u) & = -N_u = \frac{(0, a/\sqrt{1 + a^2u^2 + a^2v^2}, au(auv)'_u)}{(1 + a^2u^2 + a^2v^2)^{3/2}} = \frac{(0, a, 0)}{{(1 + a^2u^2 + a^2v^2)}^{3/2}}, \\
        W_p(\varphi_v) & = -N_v = \frac{(a/\sqrt{1 + a^2u^2 + a^2v^2}, 0, av(auv)'_v)}{(1 + a^2u^2 + a^2v^2)^{3/2}} = \frac{(a, 0, 0)}{{(1 + a^2u^2 + a^2v^2)}^{3/2}}.
    \end{align*}
    Por lo que la representación matricial de \( W_p \) en la base \( \{\varphi_u, \varphi_v\} \) es \begin{align*}
        [W_p] & = \begin{pmatrix}
                      0                                       & \frac{a}{{(1 + a^2u^2 + a^2v^2)}^{3/2}} \\
                      \frac{a}{{(1 + a^2u^2 + a^2v^2)}^{3/2}} & 0
                  \end{pmatrix}.
    \end{align*}
    Luego, \begin{align*}
        K & = \det(W_p) = -\frac{a^2}{{(1 + a^2u^2 + a^2v^2)}^{3}} \stackrel{(u,v)=(0,0)}{=} -a^2, \\
        H & = 0.
    \end{align*}
    Que es lo que queríamos demostrar. Análogamente podríamos haber definido \( \gamma(t) = (t, 0, 0) \) tal que \( \gamma(0) = (0, 0, 0) = p \) y \( \gamma'(0) = (1, 0, 0) \). Calcular \begin{align*}
        dN_{p} = \frac{d}{dt}N(\varphi(t, 0)) \Big|_{t=0} & = \frac{d}{dt}\left(\frac{(-a \cdot 0, -a t, 1)}{\sqrt{1 + a^2 t^2 + a^2 \cdot 0^2}}\right) \Big|_{t=0} = (0, -a, 0).
    \end{align*}
    Por lo que \( W_p(\gamma'(0)) = -dN_p(\gamma'(0)) = (0, a, 0) \) y obtenemos el mismo resultado. Lo mismo se puede hacer con \( \beta(t) = (0, t, 0) \), resultando en: \begin{align*}
        W_p(\beta'(0)) & = (a, 0, 0).
    \end{align*}
    Por lo tanto la representación matricial de \( W_p \) en la base \( \{\gamma'(0), \beta'(0)\} \) es la misma que antes y obtenemos los mismos valores para \( K \) y \( H \).
\end{proof}

\vspace{1em}

\begin{theorem}{3}
    Calcular la curvatura de Gauss del helicoide \( \varphi(u, v) = (v \cos(u), v \sin(u), u) \) y mostrar que es creciente en función de la distancia al eje \( z \).
\end{theorem}

\begin{proof}
    La parametrización del helicoide es \begin{align*}
        \varphi(u, v) = (v \cos(u), v \sin(u), u), \quad (u, v \in \R).
    \end{align*}
    Luego, \begin{align*}
        \varphi_u & = (-v \sin(u), v \cos(u), 1),                                                                                       \\
        \varphi_v & = (\cos(u), \sin(u), 0),                                                                                            \\
        N         & = \frac{\varphi_u \times \varphi_v}{\|\varphi_u \times \varphi_v\|} = \frac{(-\sen u, \cos u, -v)}{\sqrt{1 + v^2}}.
    \end{align*}
    Luego, la aplicación de Weingarten viene dada por: \begin{align*}
        W_p(\varphi_u) & = -N_u = \varphi_v \cdot \frac{-1}{\sqrt{1+v^2}}    \\
        W_p(\varphi_v) & = -N_v = \varphi_u \cdot \frac{1}{{(1+v^2)}^{3/2}}.
    \end{align*}
    Por lo tanto la representación matricial de \( W_p \) en la base \( \{\varphi_u, \varphi_v\} \) es \begin{align*}
        [W_p] & = \begin{pmatrix}
                      0                       & \frac{1}{{(1+v^2)}^{3/2}} \\
                      \frac{-1}{\sqrt{1+v^2}} & 0
                  \end{pmatrix}.
    \end{align*}
    Luego, \begin{align*}
        K & = \det(W_p) = -\frac{1}{{(1+v^2)}^{2}}.
    \end{align*}
    Observamos que \( K \) depende únicamente de \( v \) que es la distancia al eje \( z \) y pues \begin{align*}
        \| \varphi(u, v) - (0, 0, u) \| = |v|.
    \end{align*}
    Por lo tanto, \( K \) es creciente en función de la distancia al eje \( z \), pues \( K'(|v|) > 0 \) si consideramos a \( K \) como función de la distancia al eje \( z \).
\end{proof}

\vspace{1em}

\begin{theorem}{4}
    Mostrar que la curvatura media \( H \) en \( p \in S \) está dada por \begin{align*}
        H = \frac{1}{\pi} \int_0^{\pi} k_n(\theta) \, \mathrm{d}\theta.
    \end{align*}
    donde \( k_n(\theta) \) es la curvatura normal en la dirección que forma un ángulo \( \theta \) con la dirección principal \( e_1 \).
\end{theorem}

\begin{proof}
    En efecto, por la \textbf{Observación 4.4.4} sabemos que si \( v \in T_p S\) es unitario, \( v = e_1 \cos \theta + e_2 \sin \theta \) siendo \( \theta \) el ángulo determinado por \( v \) y \( e_1 \). Entonces \begin{align*}
        k_n(v) = k_1 \cos^2 \theta + k_2 \sin^2 \theta.
    \end{align*}
    con \( k_1, k_2 \) las curvaturas principales en \( p \). Luego, \begin{align*}
        \int_0^{\pi} k_n(\theta) \, \mathrm{d}\theta & = \int_0^{\pi} (k_1 \cos^2 \theta + k_2 \sin^2 \theta) \, \mathrm{d}\theta                                \\
                                                     & = k_1 \int_0^{\pi} \cos^2 \theta \, \mathrm{d}\theta + k_2 \int_0^{\pi} \sin^2 \theta \, \mathrm{d}\theta \\
                                                     & = k_1 \cdot \frac{\pi}{2} + k_2 \cdot \frac{\pi}{2} = \frac{\pi}{2}(k_1 + k_2).
    \end{align*}
    Por lo tanto, \begin{align*}
        \frac{1}{\pi} \int_0^{\pi} k_n(\theta) \, \mathrm{d}\theta & = \frac{1}{\pi} \cdot \frac{\pi}{2}(k_1 + k_2) = \frac{k_1 + k_2}{2} = H.
    \end{align*}
    Que es lo que queríamos demostrar.
\end{proof}

\vspace{1em}

\begin{theorem}{6}
    Mostrar que si \( H \equiv 0 \) sobre \( S \) y \( S \) no tiene puntos planares, entonces la aplicación de Gauss \( N: S \to S^2 \) satisface \begin{align*}
        \langle dN_p(w_1), dN_p(w_2) \rangle = -K(p) \langle w_1, w_2 \rangle
    \end{align*}
    para todo \( p \in S \) y todo \( w_1, w_2 \in T_p S \).
\end{theorem}

\begin{proof}
    Recordemos que \( H = \frac{1}{2} \text{tr}(W_p) \), si \begin{align*}
        W_p = \begin{pmatrix}
                  k_1 & 0   \\
                  0   & k_2
              \end{pmatrix}
    \end{align*}
    Se tiene que \( H \equiv 0 \) implica que \( k_1 + k_2 = 0 \) y por lo tanto \( k_2 = -k_1 \) y además \( k_1 \neq 0 \) pues no tiene puntos planares. Luego, la aplicación de Weingarten viene dada por \begin{align*}
        W_p = \begin{pmatrix}
                  k_1 & 0    \\
                  0   & -k_1
              \end{pmatrix}.
    \end{align*}
    Por lo tanto, para todo \( w_1, w_2 \in T_p S \) tenemos que \( K(p) = \det W_p = -k_1^2 \) y \begin{align*}
        \langle dN_p(w_1), dN_p(w_2) \rangle & = \langle -W_p(w_1), -W_p(w_2) \rangle = \langle W_p(w_1), W_p(w_2) \rangle \\
                                             & = \left \langle \begin{pmatrix}
                                                                   k_1 & 0    \\
                                                                   0   & -k_1
                                                               \end{pmatrix} w_1, \begin{pmatrix}
                                                                                      k_1 & 0    \\
                                                                                      0   & -k_1
                                                                                  \end{pmatrix} w_2 \right \rangle         \\
                                             & = k_1^2 \langle w_1, w_2 \rangle = -K(p) \langle w_1, w_2 \rangle.
    \end{align*}
    Que es lo que queríamos demostrar.
\end{proof}

\vspace{1em}

\begin{theorem}{7}
    Sea S una superficie parametrizada por \( \varphi(u, v) = (u, v u^2) \), calcular la aplicación de Weingarten \( W_p \) e indicar las curvaturas principales en un punto \( p \in S \).
\end{theorem}

\begin{proof}
    Análogamente a los ejercicios anteriores, la parametrización de la superficie es \begin{align*}
        \varphi(u, v) & = (u, v, u^2), \quad (u, v \in \R).
    \end{align*}
    Luego, \begin{align*}
        \varphi_u      & = (1, 0, 2u),                                                                                             \\
        \varphi_v      & = (0, 1, 0),                                                                                              \\
        N              & = \frac{\varphi_u \times \varphi_v}{\|\varphi_u \times \varphi_v\|} = \frac{(-2u, 0, 1)}{\sqrt{1 + 4u^2}} \\
        W_p(\varphi_u) & = -N_u = \varphi_u \cdot \frac{-2}{\sqrt{1 + 4u^2}}                                                       \\
        W_p(\varphi_v) & = -N_v = 0.
    \end{align*}
    Por lo tanto la representación matricial de \( W_p \) en la base \( \{\varphi_u, \varphi_v\} \) es \begin{align*}
        [W_p] & = \begin{pmatrix}
                      \frac{-2}{\sqrt{1 + 4u^2}} & 0 \\
                      0                          & 0
                  \end{pmatrix}.
    \end{align*}
    Luego, las curvaturas principales son \begin{align*}
        k_1 & = \frac{-2}{\sqrt{1 + 4u^2}}, \\
        k_2 & = 0.
    \end{align*}
    para todo \( p = \varphi(u, v) \in S \).
\end{proof}

\begin{theorem}{8}
    Sea S una superficie orientada. Demostrar que la suma de las curvaturas normales en un punto, en cualquier par de direcciones ortogonales, es constante.
\end{theorem}

\begin{proof}
    Sean \( u, v \in T_p S \) dos direcciones ortogonales unitarias y \( \theta \) el ángulo entre \( u \) y la dirección principal \( e_1 \). Entonces, debe ser \begin{align*}
        u & = k_1 \cos \theta + k_2 \sin \theta                                                           \\
        v & = k_1 \cos (\theta + \pi/2) + k_2 \sin (\theta + \pi/2) = -k_1 \sin \theta + k_2 \cos \theta.
    \end{align*}
    s.p.d.g. Luego, \begin{align*}
        k_n(u) & = k_1 \cos^2 \theta + k_2 \sin^2 \theta,   \\
        k_n(v) & = k_1 \sin^2 \theta + k_2 \cos^2 \theta.   \\
               & \implies k_n(u) + k_n(v) = k_1 + k_2 = 2 H
    \end{align*}
    Por lo tanto, la suma de dos curvaturas normales en un punto en direcciones ortogonales es constantemente igual a \( 2H \).
\end{proof}

\end{document}